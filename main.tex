\documentclass[a4paper,12pt]{article}

\usepackage{comment}
% Pacotes para idioma português e acentuação
\usepackage[utf8]{inputenc}
\usepackage[T1]{fontenc}
\usepackage[brazilian]{babel}

% Pacote para fonte Times New Roman
\usepackage{newtxtext,newtxmath}

% Pacotes para formatação ABNT
\usepackage{geometry}
\usepackage{titlesec}
\geometry{a4paper, top=3cm, bottom=2cm, left=3cm, right=2cm}
\usepackage{indentfirst}
\usepackage{setspace}
\setlength{\parindent}{1.25cm}
\onehalfspacing
\usepackage{caption}
\DeclareCaptionFont{custom10pt}{\fontsize{10pt}{12pt}\selectfont}
\captionsetup[figure]{font=custom10pt, justification=justified}

% Define o tamanho da fonte para todas as seções como 14pt
\titleformat*{\section}{\fontsize{14pt}{16pt}\bfseries}
\titleformat*{\subsection}{\fontsize{14pt}{16pt}\bfseries}
\titleformat*{\subsubsection}{\fontsize{14pt}{16pt}\bfseries}

% Pacotes para recursos extras
\usepackage{amsmath, amssymb, amsfonts}
\usepackage[version=4]{mhchem}
\usepackage{graphicx}
\usepackage{caption}
\usepackage{subcaption}
\usepackage{float}
\usepackage{booktabs}
\usepackage{url}
\usepackage{lipsum}
\usepackage[siunitx]{circuitikz} % pacote para fazer circuitos elétricos

% Pacote para referências no padrão ABNT
\usepackage[alf]{abntex2cite}

% Configurações para cabeçalhos e rodapés
\usepackage{fancyhdr}
\pagestyle{fancy}
\fancyhf{}
\fancyhead[R]{\thepage}

% O hyperref deve ser o último pacote carregado
\usepackage[colorlinks=true, linkcolor=black, citecolor=black, urlcolor=black]{hyperref}

\begin{document}

% Desativa a numeração das páginas para as páginas iniciais
\pagenumbering{gobble}

% Capa personalizada com titlepage
\begin{titlepage}
    \begin{center}
        \includegraphics[width=0.15\textwidth]{Brasão_da_UEFS.png}\\[1.5cm]
        
        % Faculdade e curso
        {\fontsize{18pt}{20pt}\bfseries UNIVERSIDADE ESTADUAL DE FEIRA DE SANTANA}\\[0.3cm]
        {\fontsize{18pt}{20pt}\bfseries Engenharia de Computação}\\[2cm]


        % Alunos
        {\fontsize{14pt}{16pt}\bfseries\itshape Aluno 1}\\[0.1cm]
        {\fontsize{14pt}{16pt}\bfseries\itshape Aluno 2}\\[0.1cm]
        {\fontsize{14pt}{16pt}\bfseries\itshape Aluno 3}\\[0.1cm]
        {\fontsize{14pt}{16pt}\bfseries\itshape Aluno 4}\\[3cm]

        
        % Título do experimento
        {\fontsize{18pt}{20pt}\bfseries Experimento X: Medida da Componente Horizontal do Campo Magnético da Terra}\\[2cm]

        
        % Cidade e ano no rodapé da página
        \vfill
        {\fontsize{12pt}{14pt}\bfseries Feira de Santana - BA}\\
        {\fontsize{12pt}{14pt}\bfseries \the\year}
    \end{center}
\end{titlepage}

% Contra-capa
\begin{titlepage}
    \begin{center}
        % Alunos
        {\fontsize{14pt}{16pt}\bfseries\itshape Aluno 1}\\[0.1cm]
        {\fontsize{14pt}{16pt}\bfseries\itshape Aluno 2}\\[0.1cm]
        {\fontsize{14pt}{16pt}\bfseries\itshape Aluno 3}\\[0.1cm]
        {\fontsize{14pt}{16pt}\bfseries\itshape Aluno 4}\\[3cm]

        % Título do Experimento
        {\fontsize{18pt}{20pt}\bfseries Experimento X: Medida da Componente Horizontal do Campo Magnético da Terra}\\[6cm]

       \begin{flushright}
        \begin{minipage}{0.5\textwidth} % controla a largura e faz o texto "andar para a direita"
        \onehalfspacing
        \fontsize{12pt}{15pt}\selectfont
        \justifying
        Relatório apresentado como atividade avaliativa parcial da disciplina de Física Experimental III, pertencente ao curso de Engenharia de Computação da Universidade Estadual de Feira de Santana (UEFS).\\[0.1cm]
        Docente: Prof. Ernando Silva Ferreira.
    \end{minipage}
    \end{flushright}

        % Cidade e ano no rodapé da página
        \vfill
        {\fontsize{12pt}{14pt}\bfseries Feira de Santana - BA}\\
        {\fontsize{12pt}{14pt}\bfseries \the\year}
    \end{center}
\end{titlepage}

% Sumário
\tableofcontents
\newpage

% Começa a numeração a partir do resumo
\pagenumbering{arabic}

% Resumo
\section{RESUMO}

\lipsum[1]

\newpage
% Introdução
\section{INTRODUÇÃO}
\label{sec:introducao}

\lipsum[1-3]

\newpage
% Desenvolvimento
\section{DESENVOLVIMENTO}
\label{sec:desenvolvimento}

\lipsum[1]

\subsection{Discussão Teórica}
\label{subsec:discussao_teorica}

\lipsum[1-3]

\subsection{Procedimentos Experimentais}
\label{subsec:procedimentos_experimentais}

\lipsum[1-2]

\subsection{Tratamento de Dados e Resultados}
\label{subsec:discussao_teorica}

\lipsum[1-2]

\newpage
% Conclusão
\section{CONCLUSÃO}
\label{sec:conclusao}

\lipsum[1-2]

\begin{figure}[H]
\centering
\caption{Imagem exemplo}
\includegraphics[width=0.6\textwidth]{experimento.jpg}
\label{fig:carga_capacitor}
\par\vspace{0.5em}
{\footnotesize\textbf{Fonte:} Elaborada pelos autores (2025)}
\end{figure}


% Referências
\newpage
\addcontentsline{toc}{section}{Referências}  % Adiciona as referências ao TOC

\begin{thebibliography}{99}

\bibitem{halliday2016}
HALLIDAY, David; RESNICK, Robert; WALKER, Jearl. 
\textit{Fundamentos de Física, volume 3: Eletromagnetismo}, 10ª edição. 
Rio de Janeiro: LTC, 2016.

\bibitem{young2015}
YOUNG, Hugh D.; FREEDMAN, Roger A. 
\textit{Física III: Eletromagnetismo}, 14ª edição. 
São Paulo: Pearson, 2015.

\bibitem{vuolo1996}
VUOLO, José H. 
\textit{Fundamentos de Teoria dos Erros}, 2ª edição.
São Paulo: Editora Blucher, 1996.

\end{thebibliography}


\end{document}